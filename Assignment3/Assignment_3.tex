\documentclass[journal,12pt,twocolumn]{IEEEtran}
%
\usepackage{setspace}
\usepackage{gensymb}
\usepackage{siunitx}
\usepackage{tkz-euclide} 
\usepackage{textcomp}
\usepackage{standalone}
\usetikzlibrary{calc}
\newcommand\hmmax{0}
\newcommand\bmmax{0}

%\doublespacing
\singlespacing

%\usepackage{graphicx}
%\usepackage{amssymb}
%\usepackage{relsize}
\usepackage[cmex10]{amsmath}
%\usepackage{amsthm}
%\interdisplaylinepenalty=2500
%\savesymbol{iint}
%\usepackage{txfonts}
%\restoresymbol{TXF}{iint}
%\usepackage{wasysym}
\usepackage{amsthm}
%\usepackage{iithtlc}
\usepackage{mathrsfs}
\usepackage{txfonts}
\usepackage{stfloats}
\usepackage{bm}
\usepackage{cite}
\usepackage{cases}
\usepackage{subfig}
%\usepackage{xtab}
\usepackage{longtable}
\usepackage{multirow}
%\usepackage{algorithm}
%\usepackage{algpseudocode}
\usepackage{enumitem}
\usepackage{mathtools}
\usepackage{steinmetz}
\usepackage{tikz}
\usepackage{circuitikz}
\usepackage{verbatim}
\usepackage{tfrupee}
\usepackage[breaklinks=true]{hyperref}
%\usepackage{stmaryrd}
\usepackage{tkz-euclide} % loads  TikZ and tkz-base
%\usetkzobj{all}
\usetikzlibrary{calc,math}
\usepackage{listings}
    \usepackage{color}                                            %%
    \usepackage{array}                                            %%
    \usepackage{longtable}                                        %%
    \usepackage{calc}                                             %%
    \usepackage{multirow}                                         %%
    \usepackage{hhline}                                           %%
    \usepackage{ifthen}                                           %%
  %optionally (for landscape tables embedded in another document): %%
    \usepackage{lscape}     
\usepackage{multicol}
\usepackage{chngcntr}
\usepackage{amsmath}
\usepackage{cleveref}
\usepackage{amsmath}
%\usepackage{enumerate}

%\usepackage{wasysym}
%\newcounter{MYtempeqncnt}
\DeclareMathOperator*{\Res}{Res}
%\renewcommand{\baselinestretch}{2}
\renewcommand\thesection{\arabic{section}}
\renewcommand\thesubsection{\thesection.\arabic{subsection}}
\renewcommand\thesubsubsection{\thesubsection.\arabic{subsubsection}}

\renewcommand\thesectiondis{\arabic{section}}
\renewcommand\thesubsectiondis{\thesectiondis.\arabic{subsection}}
\renewcommand\thesubsubsectiondis{\thesubsectiondis.\arabic{subsubsection}}

% correct bad hyphenation here
\hyphenation{op-tical net-works semi-conduc-tor}
\def\inputGnumericTable{}                                 %%

\lstset{
%language=C,
frame=single, 
breaklines=true,
columns=fullflexible
}
%\lstset{
%language=tex,
%frame=single, 
%breaklines=true
%}
\usepackage{graphicx}
\usepackage{pgfplots}

\begin{document}


\newtheorem{theorem}{Theorem}[section]
\newtheorem{problem}{Problem}
\newtheorem{proposition}{Proposition}[section]
\newtheorem{lemma}{Lemma}[section]
\newtheorem{corollary}[theorem]{Corollary}
\newtheorem{example}{Example}[section]
\newtheorem{definition}[problem]{Definition}
%\newtheorem{thm}{Theorem}[section] 
%\newtheorem{defn}[thm]{Definition}
%\newtheorem{algorithm}{Algorithm}[section]
%\newtheorem{cor}{Corollary}
\newcommand{\BEQA}{\begin{eqnarray}}
\newcommand{\EEQA}{\end{eqnarray}}
\newcommand{\define}{\stackrel{\triangle}{=}}
\bibliographystyle{IEEEtran}
%\bibliographystyle{ieeetr}
\providecommand{\mbf}{\mathbf}
\providecommand{\abs}[1]{\ensuremath{\left\vert#1\right\vert}}
\providecommand{\norm}[1]{\ensuremath{\left\lVert#1\right\rVert}}
\providecommand{\mean}[1]{\ensuremath{E\left[ #1 \right]}}
\providecommand{\pr}[1]{\ensuremath{\Pr\left(#1\right)}}
\providecommand{\qfunc}[1]{\ensuremath{Q\left(#1\right)}}
\providecommand{\sbrak}[1]{\ensuremath{{}\left[#1\right]}}
\providecommand{\lsbrak}[1]{\ensuremath{{}\left[#1\right.}}
\providecommand{\rsbrak}[1]{\ensuremath{{}\left.#1\right]}}
\providecommand{\brak}[1]{\ensuremath{\left(#1\right)}}
\providecommand{\lbrak}[1]{\ensuremath{\left(#1\right.}}
\providecommand{\rbrak}[1]{\ensuremath{\left.#1\right)}}
\providecommand{\cbrak}[1]{\ensuremath{\left\{#1\right\}}}
\providecommand{\lcbrak}[1]{\ensuremath{\left\{#1\right.}}
\providecommand{\rcbrak}[1]{\ensuremath{\left.#1\right\}}}
\theoremstyle{remark}
\newtheorem{rem}{Remark}
\newcommand{\sgn}{\mathop{\mathrm{sgn}}}
\providecommand{\res}[1]{\Res\displaylimits_{#1}} 
%\providecommand{\norm}[1]{\lVert#1\rVert}
\providecommand{\mtx}[1]{\mathbf{#1}}
\providecommand{\fourier}{\overset{\mathcal{F}}{ \rightleftharpoons}}
%\providecommand{\hilbert}{\overset{\mathcal{H}}{ \rightleftharpoons}}
\providecommand{\system}{\overset{\mathcal{H}}{ \longleftrightarrow}}
	%\newcommand{\solution}[2]{\textbf{Solution:}{#1}}
\newcommand{\solution}{\noindent \textbf{Solution: }}
\newcommand{\cosec}{\,\text{cosec}\,}
\providecommand{\dec}[2]{\ensuremath{\overset{#1}{\underset{#2}{\gtrless}}}}
\newcommand{\myvec}[1]{\ensuremath{\begin{pmatrix}#1\end{pmatrix}}}
\newcommand{\mydet}[1]{\ensuremath{\begin{vmatrix}#1\end{vmatrix}}}
%\numberwithin{equation}{section}
\numberwithin{equation}{subsection}
%\numberwithin{problem}{section}
%\numberwithin{definition}{section}
\makeatletter
\@addtoreset{figure}{problem}
\makeatother
\let\StandardTheFigure\thefigure
\let\vec\mathbf
%\renewcommand{\thefigure}{\theproblem.\arabic{figure}}
\renewcommand{\thefigure}{\theproblem}
%\setlist[enumerate,1]{before=\renewcommand\theequation{\theenumi.\arabic{equation}}
%\counterwithin{equation}{enumi}
%\renewcommand{\theequation}{\arabic{subsection}.\arabic{equation}}
\def\putbox#1#2#3{\makebox[0in][l]{\makebox[#1][l]{}\raisebox{\baselineskip}[0in][0in]{\raisebox{#2}[0in][0in]{#3}}}}
     \def\rightbox#1{\makebox[0in][r]{#1}}
     \def\centbox#1{\makebox[0in]{#1}}
     \def\topbox#1{\raisebox{-\baselineskip}[0in][0in]{#1}}
\vspace{3cm}
\title{Polynomial Curve Fitting}
\maketitle
\newpage
%\tableofcontents
\bigskip
\renewcommand{\thefigure}{\theenumi}
\renewcommand{\thetable}{\theenumi}
\begin{abstract}
This document contains theory involved in curve fitting.
\end{abstract}
\section{\textbf{Objective}}
The objective is to fit best line for the polynomial curve using regularization.
\section{Generate Dataset}
Create a sinusoidal function of the form
\begin{align}
    y = A\sin{2\pi x} + n(t) \label{eq:1}
\end{align}
n(t) is the random noise that is included in the training set. This set consists of N samples of input data i.e. x expressed as shown below
\begin{align}
    x = \myvec{x_{1}, x_{2}, .., x_{N}}^{T}
\end{align}
which give the corresponding values of y denoted as
\begin{align}
    y = \myvec{y_{1}, y_{2}, .., y_{N}}^{T}
\end{align}
\begin{figure}[!h]
\begin{center}
\includegraphics[width=3.4in]{figs/fig1.png}
\end{center}
\caption{Sinusoidal Dataset with added noise}
\label{fig:1}
\end{figure}
The Fig \ref{fig:1} is generated by random values of $x_{n}$ for n =1,2,..,N.
where N=50 in the range [0,1].

The corresponding values of y were generated from the Eq \eqref{eq:1}.The first term $A\sin{2\pi x}$ was computed directly and then random noise samples having a normal(Gaussian) distribution are added inorder to get the corresponding values of y.
\begin{lstlisting}
#Generate the sine curve 
import numpy as np
import matplotlib.pyplot as plt

N = 50
np.random.seed(20)
x = np.sort(np.random.rand(N,1),axis=0)
noise = np.random.normal(0,0.3,size=(N,1))
A = 2.5
y = A*np.sin(2*np.pi*x) + noise

plt.scatter(x,y,c='b',marker='o',label='Data with noise')
plt.xlabel('x');plt.ylabel('y')
\end{lstlisting}
%
\section{Polynomial Curve Fitting}
The goal is to find the best line that fits into the  pattern of the training data shown in the graph. By using regularization, we add a penalty term to the error function inorder to discourage the coefficients from reaching large values.
We shall fit the data using a polynomial function of the form, 
\begin{align}
     y\brak{w,x}= \myvec{w_0 & w_1 & w_2 &.&...w_N}\myvec{1 \\x\\x^2\\.\\x^N}= \sum_{j=0}^{M} w_j x^{j}
\end{align}
where M is the order of the polynomial.
The polynomial coefficient are collectively denoted by the vector w.
Although the polynomial function y$\brak{w,x}$ is a nonlinear function of x, it
is a linear function of the coefficients $\textbf{w}$. Functions, such as the polynomial, which are linear in the unknown parameters have important properties and are called \textbf{linear models}.
\section{Training the Model}
First, we take a random guess of the sine curve on the input data
\begin{lstlisting}
plt.plot(np.linspace(0,1,50),np.sin(2*np.linspace(0,1,50)*np.pi),c='g',linewidth=2,label='function generating input data')
\end{lstlisting} 
Here, we use regularization parameter $\lambda$ which is used to control the overfitting phenomenon and initialize the polynomial degree. 
\begin{lstlisting}
lamda = 0.00000001522 
poly_deg = 3
\end{lstlisting}
Now we generate the input matrix f which is obtained from the random values of x.
\begin{lstlisting}
f = np.zeros(shape = (N,poly_deg+1))
f[:,0] = 1
for i in range(1,poly_deg+1):
    f[:,i] = np.power(x,i).reshape((N,))    
\end{lstlisting}
Now, we use regularization and fit our model with all the parameters and function to generate the plot.
\begin{lstlisting}
W = np.linalg.pinv((f.T.dot(f) + lamda*np.eye(poly_deg+1))).dot(f.T).dot(y)
\end{lstlisting}
You need to vary $\lambda$ to make it work.
\begin{figure}[!h]
\begin{center}
\includegraphics[width=3.4in]{figs/fig2.png}
\end{center}
\caption{Curve Fitting for ln$\lambda = -18$}
\label{fig:2}
\end{figure}
\begin{figure}[!h]
\begin{center}
\includegraphics[width=3.4in]{figs/fig3.png}
\end{center}
\caption{Curve Fitting for ln$\lambda = 0$}
\label{fig:3}
\end{figure}
\section{Curve fitting using scikit-learn}
scikit learn is a machine learning python library which features various algorithms and is designed to interoperate with the numpy and scipy libraries.
Here, we import and use linear regression to find the best fit.
\begin{lstlisting}
from sklearn.preprocessing import PolynomialFeatures
sk_poly_deg=3
poly_feature = PolynomialFeatures(degree=sk_poly_deg,include_bias=False)
x_poly = poly_feature.fit_transform(x)

from sklearn.linear_model import LinearRegression
lin_reg=LinearRegression()
lin_reg.fit(x_poly,y)
\end{lstlisting}
\begin{figure}[!h]
\begin{center}
\includegraphics[width=3.4in]{figs/fig4.png}
\end{center}
\caption{Using scikit-learn}
\label{fig:4}
\end{figure}
Python code:
\begin{lstlisting}
https://github.com/Hrithikraj2/EE4015_IDP/blob/main/Assignment3/Assignment_3_Regularization.ipynb
\end{lstlisting}
\end{document}
